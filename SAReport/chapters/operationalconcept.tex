We questioned which requirement could be critical for a Situational Awareness System. This phase involved identifying and understanding the essential needs that the system must fulfill to be effective and user-friendly. 

To achieve this, we conducted a detailed analysis of the \textit{Operational Concept}, which involved the development of detailed personas and scenario analysis. 

The Operational Concept is a critical phase in the design of complex systems, translating system requirements into actionable plans. 
It provides a comprehensive understanding of the system's operational environment, including the roles and responsibilities of the users, the tasks they perform, and the context in which they operate.

Since we are developing an e-learning dashboard intended for users who may access it from various locations, we have not defined specific environmental constriants. The flexibility of online learning environments means that users could be utilizing the system in diverse settings, such as homes, offices, or public spaces, each with varying levels of connectivity and hardware capabilities.

\section{Scenario}
The company requires all employees to pursue upskilling or reskilling opportunities based on their previous studies and work experiences. Marta and Matteo are two employees whose skills and knowledge will be monitored to ensure they can effectively contribute to projects in the Offensive Cybersecurity field. The company’s e-learning platform provides courses that enable employees to obtain the \textit{OSCP} (Offensive Security Certified Professional) certification.

The platform is accessible both on-site at the company and remotely, offering flexible learning options that accommodate diverse schedules. Each employee receives a personalized dashboard where they can track their progress and access a wide range of educational resources, including video courses (concept pills), slide decks, and practical exercises. At the end of each module, employees can take assessments to reinforce their understanding and ensure they meet the learning objectives.

Additionally, the platform includes analytics and trend reports that help employees understand their learning journey, estimate the time required for course completion, and measure their engagement levels. The course design features a tailored approach, continuously monitoring each employee’s performance and adapting to their specific learning needs. This customization enhances the overall learning experience, ensuring employees are well-prepared for the OSCP certification.

\section{Supported Figure: Matteo}
Matteo is a student from the University of Salerno, with a bachelor's degree in Computer Engineering. His passion for cybersecurity has led him to specialize in this field, acquiring basic skills ranging from understanding the fundamentals of cybersecurity to networking and network protocols, from vulnerability analysis to familiarity with essential security tools. Now, Matteo faces a new challenge: a project in collaboration with Marta concerning Offensive Cybersecurity. However, to best address this task, Matteo needs to broaden his skills and knowledge.

\begin{center}
    \begin{tabular}{|l|p{10cm}|}
    \hline
    \textbf{Characteristic} & \textbf{Description} \\
    \hline
    Age Range & 20-25 \\
    \hline
    Gender & Male \\
    \hline
    Culture & Italian \\
    \hline
    Education & Bachelor's degree in Computer Engineering \\
    \hline
    Language & Italian, English (proficient for technical literature) \\
    \hline
    Frequency of Use & Several times a week \\
    \hline
    Experience & Familiar with basic cybersecurity tools and platforms, intermediate programming skills \\
    \hline
    Personality & Curious, analytical, detail-oriented, enjoys problem-solving, goal-oriented, collaborative \\
    \hline
    Acquired Skills & Fundamentals of cybersecurity, networking, vulnerability analysis, basic scripting/programming, offensive cybersecurity \\
    \hline
    Learning Style & Visual and hands-on learner \\
    \hline
    \end{tabular}
    \end{center}
    
\newpage
\section{Supported Figure: Marta}
Marta is a student from the University of Naples Federico II. She completed a bachelor's degree in Computer Engineering and has now specialized in machine learning, acquiring basic skills in the field of artificial intelligence, including the structure and applications of neural networks and deep neural networks, their applications in robotics, and autonomous driving. Marta wants to collaborate with Matteo on a new project in the field of Offensive Cybersecurity. Since Marta has followed a different academic path from Matteo's, which does not involve cybersecurity, she needs to upskill her competencies.

\begin{center}
    \begin{tabular}{|l|p{10cm}|}
    \hline
    \textbf{Characteristic} & \textbf{Description} \\
    \hline
    Age Range & 20-25 \\
    \hline
    Gender & Female \\
    \hline
    Culture & Italian \\
    \hline
    Education & Bachelor's degree in Computer Engineering, specializing in Machine Learning \\
    \hline
    Language & Italian, English (proficient for technical literature) \\
    \hline
    Frequency of Use & A few times a week \\
    \hline
    Experience & Skilled in AI and Machine Learning platforms, novice in cybersecurity \\
    \hline
    Personality & Innovative, inquisitive, enjoys learning new skills, collaborative, adaptable \\
    \hline
    Acquired Skills & Basics of AI, neural networks, deep learning, robotics, autonomous driving, basic programming, willingness to learn cybersecurity \\
    \hline
    Learning Style & Visual and auditory learner, prefers structured guidance \\
    \hline
    \end{tabular}
    \end{center}