The objective of this project is to implement a comprehensive e-learning system designed to assist users 
in tailoring their learning pathways and enhancing their skill sets. Developed with a strong emphasis on 
situational awareness, this project is structured into two primary components: the first involves 
\textit{Goal-Driven Task Analysis} (GDTA), and the second focuses on the implementation of an integrated dashboard.

This system is meticulously crafted with a \textbf{User-Centric Approach}, ensuring that users can effectively 
monitor their competencies and track their learning progress. In developing the GDTA, we delineated operational 
concepts through detailed personas and scenario analysis, ensuring that the system is attuned to the diverse needs 
and contexts of its users. To fulfill the goals described inside the GDTA, we implemented two dashboards: the first
dashboard is focused on the user's global learning progress, while the second dashboard is focused on the user's
progression in a specific course.

For the dashboard implementation, we leveraged \textit{ElasticSearch} as the underlying database and \textit{Kibana} 
as the visualization tool. The data presented on the dashboard were specifically curated for this project and are 
stored in multiple .csv files. 

Initially, our goal was to conduct a comprehensive data analysis using an extensive dataset found online. However, 
we faced significant challenges in locating suitable data that aligned perfectly with our needs. Consequently, we 
chose to create our own .csv files containing the essential data required to populate our dashboard. 
This decision granted us full control over the data, enabling precise customization according to the specific 
requirements of our project. By curating our own dataset, we ensured the accuracy and relevance of the information 
presented, thereby facilitating a more insightful and effective implementation of our e-learning system. The data 
are shown on a time slice of twenty days, from the first of May to the twentieth of May. The reason for 
this time interval is to illustrate the user's learning situation on the platform before the end of the courses,
which is set for the last day of May. 

