This paragraph will present the two dashboards implemented, the design principles used, and the SA demons that were tried to avoid. As mentioned before, to ensure that most of the requirements identified in the previous stage could be represented, it was decided to use a dataset entirely created by us to have full control over the data and its structure.

The whole system is divided into two dashboards, each associated with one of the goals presented before.

\begin{itemize}
    \item \textbf{Dashboard I}: Subgoal 1.2.1 - Improve student engagement with the platform and Subgoal 1.2.2 - Optimize student learning through personalization.
    \item \textbf{Dashboard II}: Subgoal 1.1.1 - Comprehension of the fundamental concepts of cybersecurity.
\end{itemize}

Since creating a dashboard for each subgoal seemed too limiting, we decided to implement the first dashboard by incorporating the requirements of \textit{Subgoals 1.2.1 and 1.2.2}. 

For the second dashboard, we found it more appropriate to integrate the informational requirements of \textit{Subgoals 1.2.1 and 1.2.2}. This is because the informational requirements of \textit{Subgoal 1.1} are solely focused on ensuring the understanding of topics across different courses. In order to avoid redundancy in informational requirements, we chose to include those of \textit{Subgoals 1.2.1 and 1.2.2} within the dashboard dedicated to a specific course—in our case, \textbf{Cybersecurity Fundamentals}. 





