To ensure that all identified requirements were met, it was decided, 
as previously mentioned, to use a dataset entirely created by us. This decision was driven by the need to have 
complete control over the variables and information present, thus ensuring that every required aspect was 
accurately represented and manageable. The dataset created consists of different CSV files, each one representing a table. 
These files describe the courses within an e-learning platform, the interactions between a user and courses,
the interactions between users, the resources used by the user on the platform, and the user's
results in the courses.


The dataset is composed of the following tables:
\begin{itemize}
    \item 
        \textbf{Completed Course:} This table indicates whether the user has completed something each 
        day or not for the different courses;
    \item 
        \textbf{Completed Modules:} This table indicates whether the user has completed something each 
        day or not for the different modules;
    \item 
        \textbf{User Hours Sessions:} This table represents the hours spent by the user on the platform 
        each day and the average hours spent by other users;
    \item 
        \textbf{User Hours Sessions per Module:} This table represents the hours spent by the user for each module
        of a course each day and the average hours spent by other users;
    \item 
        \textbf{Resources:} This table represents the hours of use of the different resources that are
        available on the platform for each day;
    \item 
        \textbf{Resources per Course:} This table represents the hours of use of the different resources that are
        available for a course;
    \item 
        \textbf{Forum:} This table represents the interactions between users on the forum, showing the number
        of questions asked, the number of answers given, and the number of answers received;
    \item 
        \textbf{Forum per Module:} This table represents the interactions between users on the forum for each module of a course, showing the number
        of questions asked, the number of answers given, and the number of answers received;
    \item 
        \textbf{Completed Course Votes:} This table shows the votes obtained by the user for each completed test of the courses, 
        categorized by type (theory or exercise);
    \item 
        \textbf{Completed Modules Votes:} This table shows the votes obtained by the user for each completed module, 
        categorized by type (theory or exercise).
\end{itemize}