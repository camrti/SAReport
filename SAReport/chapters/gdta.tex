We applied the \textit{Cognitive Task Analysis} (CTA) methodology to explore individuals' knowledge and thought processes. Among the various CTA methodologies available, we selected \textit{Goal-Directed Task Analysis} (GDTA), specifically focusing on Situation Awareness (SA) and the goals inherent in SA processes.

GDTA is tailored to identify user goals, the decisions they make, and the critical information needed to achieve these goals. This methodology not only maps out user objectives but also provides insights into their decision-making strategies and information requirements.

By employing GDTA, our aim is to deepen our understanding of how users perceive and interact with information. This understanding informs our design process, ensuring that our solutions align closely with user needs and preferences. Ultimately, this approach enhances the usability and effectiveness of our project outcomes by prioritizing User-Centric design principles.

\section{Initial GDTA Goal Tree}
The initial GDTA Goal Tree, shown in Figure 4.1, identified three Major-Goals. However, after an in-depth review and analysis, we concluded that the Major-Goal 1 \textbf{"Identify prior knowledge in the field of Cybersecurity"} can be integrated into the Major-Goal 2 \textbf{"Define and evaluate the knowledge required for Offensive Cybersecurity by monitoring students' progress"}. This conclusion was drawn from a comprehensive analysis of the sub-goals associated with Major-Goal 1.

Upon examining the sub-goals of Major-Goal 1, it became evident that both the sub-goals and the Major-Goal itself address topics that students need to master throughout their learning journey in cybersecurity. Given that Major-Goal 2 aims to delineate the specific topics and skills students must acquire, we decided to incorporate Major-Goal 1 as a sub-goal within Major-Goal 2. This integration ensures a more cohesive framework for evaluating and defining the necessary knowledge for Offensive Cybersecurity.

Furthermore, the sub-goals of Major-Goal 1 have been reclassified as Level 2 elements, signifying their importance in the comprehension phase of the learning process. 

\begin{figure}[H]
    \centering
    \includegraphics[width=\textwidth]{./assets/initialgdta.png}
    \caption{Initial GDTA Goal Tree}
    \label{fig:Initial GDTA}
\end{figure}

\newpage
\section{Final GDTA Goal Tree}
In Figure 4.2, we present our Final GDTA Goal Tree, which outlines the primary goals of the system and the sub-goals that contribute to their achievement. In particular, our GDTA Goal Tree supports users in adapting their learning paths to enhance their expertise in Offensive Cybersecurity.
The \textit{Overall Operator Goal} is broken down into two primary \textit{Major-Goals} as shown in the picture below. Each Major-Goal is further divided into \textit{Sub-Goals} that are essential for achieving the Major-Goals. 

\begin{figure}[H]
    \centering
    \includegraphics[width=\textwidth]{./assets/GDTA.png}
    \caption{Final GDTA Goal Tree}
    \label{fig:GDTA}
\end{figure}

The following sections will delve into the details of each sub-goal, providing a comprehensive understanding of the cognitive processes involved in achieving these objectives, rather than focusing on the methods or actions needed.
Achieving these goals necessitates more complex cognitive processes than simply searching for a single piece of information.

Subsequently, we defined the \textit{Informational Requirements} for each goal, which are crucial for users to make informed decisions and achieve their objectives. While these requirements might be mistakenly viewed as lower-level goals within the hierarchy, they actually serve as supportive elements that facilitate the achievement of primary goals.

In order to fulfill a goal, a decision must be made based on the available information. We are not interested in trivial yes-or-no questions: instead, we are focusing on decisions that enable the fulfillment of high-level goals. Moreover, these decisions require complex cognitive processes and a deep understanding of the situation.

\newpage
\section{Major-Goal 1.1: Ensure the correct comprehension of the necessary knowledge for Offensive Cybersecurity}
In this section, we describe Major Goal 1.1 and its sub-goals.
The main aim of this goal, along with its sub-goals, is to outline the modules that users need to study to enhance their skills. Essentially, this major goal represents how the course is structured in terms of modules that every student on the platform must learn.

We've taken a different approach compared to the typical GDTA (Goal Directed Task Analysis) process. The specific problem we're addressing (e-learning) isn't well-suited to the GDTA techniques typically taught in class. Therefore, we explain how we've proposed sub-goals and levels of perception, comprehension, and projection for these sub-goals.

The objective of this major goal is to enable all platform users to acquire Offensive Cybersecurity knowledge. Although the sub-goals may initially seem like tasks, they are intended as individual objectives that each student must achieve to solidify their understanding of the field. 

There are no time constraints for achieving these individual objectives, allowing students to start studying whichever module they prefer and proceed at their own pace. This approach enables users to create a personalized learning path: they can skip modules they already know and only take end-of-module tests, or revisit topics where they feel they need more practice based on platform recommendations.

Regarding \textit{Perception Level}, we've included the fundamental concepts of each course that students perceive, focusing on concise micro-learning. For instance, topics such as firewall, PHP, and SHA256 are covered in concept pills.

At the \textit{Comprehension Level}, we've incorporated broader concepts that connect one or more micro-learning modules. The goal is to help students understand concepts that emerge after assimilating micro-learning, such as data integrity through understanding algorithms like MAC combined with SHA256.

Finally, at the \textit{Projection Level}, we've identified the advanced competencies that students acquire through comprehension and perception, such as the ability to apply learned methodologies in future scenarios.


\newpage
\subsection{Sub-Goal 1.1.1: Comprehension of the fundamental concepts of cybersecurity }
The objective of comprehending the fundamental concepts of cybersecurity is to provide individuals with a robust understanding of essential security principles and practices. Key areas of focus include ensuring confidentiality through mechanisms like OTP and MAC, maintaining data integrity with tools such as SHA256, and guaranteeing availability via digital signatures. Additionally, an understanding of blockchain technology and threat models is crucial, along with proficiency in cybersecurity algorithms and protocols like TLS. This foundational knowledge enables individuals to critically evaluate emerging technologies in relation to IT security and stay informed about current laws and regulations.
The decision associated with subgoal 1.1.1 and its SA requirements are shown in the following figures.
\begin{figure}[H]
    \centering
    \includegraphics[width=\textwidth]{./assets/subgoal_1.1.1.png}
    \caption{Sub-Goal 1.1.1}
    \label{fig:subgoal_1.1.1}
\end{figure}

\begin{table}[H]
    \begin{center}
    \begin{tabular}{ | m{5cm} | m{5cm}| m{5cm} | } 
      \hline
      \textbf{Level 1 SA requirements} & \textbf{Level 2 SA requirements}  & \textbf{Level 3 SA requirements}  \\ 
      \hline
      OTP & Confidentiality & Capability to critically evaluate emerging technologies in relation to IT security, current laws and regulations\\ 
      \hline
      SHA256 & Integrity & \\ 
      \hline
      MAC & Confidentiality & \\ 
      \hline
      Digital Signature & Availability  & \\ 
      \hline
      Blockchain & Threat Models  & \\ 
      \hline
      TLS Protocol & Algorithms for Cybersecurity & \\ 
      \hline
    \end{tabular}
    \end{center}
    \caption{SA requirements for subgoal 1.1.1}
    \end{table}
    
\newpage
\subsection{Sub-Goal 1.1.2: Ensure a thorough understanding of Network Penetration Testing}
The objective of ensuring a thorough understanding of network penetration testing is to equip individuals with the necessary skills and knowledge to effectively identify and mitigate security vulnerabilities within network infrastructures. This includes mastering fundamental programming concepts such as Python operators, syntax, and Powershell scripting, as well as understanding key network protocols like the Internet Protocol (IP) and Domain Name System (DNS). Advanced competencies involve applying cryptography techniques and hashing, acquiring in-depth Windows networking knowledge, and efficiently using variables, loops, and functions in Python and Powershell. Ultimately, this comprehensive understanding enables individuals to select the most appropriate penetration testing strategies for various scenarios, ensuring robust network security.
The decision associated with subgoal 1.1.2 and its SA requirements are shown in the following figures.
\begin{figure}[H]
    \centering
    \includegraphics[width=\textwidth]{./assets/subgoal_1.1.2.png}
    \caption{Sub-Goal 1.1.2}
    \label{fig:subgoal_1.1.2}
\end{figure}

\begin{table}[H]
    \begin{center}
    \begin{tabular}{ | m{5cm} | m{5cm}| m{5cm} | } 
      \hline
      \textbf{Level 1 SA requirements} & \textbf{Level 2 SA requirements}  & \textbf{Level 3 SA requirements}  \\ 
      \hline
      Python operators & Cryptography techniques and hashing & Capability to choose the best penetration testing strategies based on the situation\\ 
      \hline
      Python syntax & Windows networking knowledge & \\ 
      \hline
      Powershell Scripting & Usage of variables & \\ 
      \hline
      Internet Protocol & Loops and functions in Python and Powershell  & \\ 
      \hline
      Domain Name System &  & \\ 
      \hline
    \end{tabular}
    \end{center}
    \caption{SA requirements for subgoal 1.1.2}
    \end{table}

\newpage
\subsection{Sub-Goal 1.1.3: Promote a solid understanding of SOC pratices and processes}
The objective of promoting a solid understanding of Security Operations Center (SOC) practices and processes is to provide individuals with the knowledge and skills necessary to effectively manage and respond to cybersecurity incidents. This includes foundational knowledge of network protocols such as the Internet Protocol (IP) and Domain Name System (DNS), as well as scripting skills with Powershell. Advanced competencies involve data conversion in Python between decimal, binary, and hexadecimal formats, understanding operational security and security management, and familiarity with practices such as the Cyber Kill Chain and logging. Ultimately, this comprehensive understanding enables individuals to proficiently detect and respond to cyber threats, ensuring robust security operations.
The decision associated with subgoal 1.1.3 and its SA requirements are shown in the following figures.

\begin{figure}[H]
    \centering
    \includegraphics[width=\textwidth]{./assets/subgoal_1.1.3.png}
    \caption{Sub-Goal 1.1.3}
    \label{fig:subgoal1.1.3}
\end{figure}

\begin{table}[H]
\begin{center}
\begin{tabular}{ | m{5cm} | m{5cm}| m{5cm} | } 
  \hline
  \textbf{Level 1 SA requirements} & \textbf{Level 2 SA requirements}  & \textbf{Level 3 SA requirements}  \\ 
  \hline
  Internet Protocol & Data conversion in Python between decimal, binary, and hexadecimal & Knowing how to detect and respond to cyber threats\\ 
  \hline
  Powershell Scripting & knowledge of operational security and security management & \\ 
  \hline
  Domain Name System & practices of Cyber Kill Chain and Logging & \\ 
  \hline
  Firewall &  & \\ 
  \hline
\end{tabular}
\end{center}
\caption{SA requirements for subgoal 1.1.3}
\end{table}

\newpage
\subsection{Sub-Goal 1.1.4: Acquire advanced skills in Web Application Essentials}
The objective of acquiring advanced skills in web application essentials is to enable individuals to develop and maintain secure web applications. This encompasses foundational knowledge of web development technologies such as HTML, CSS, PHP, and JavaScript, along with proficiency in security tools like ZAP, AFL, SonarQube, and Flawfinder. Advanced skills include managing secure sessions, handling authentication, authorization, passwords, and cookies, and ensuring the security of REST, SOAP, and GraphQL services, as well as security practices in GIT. Ultimately, this comprehensive understanding equips individuals to recognize and mitigate vulnerabilities such as Server Side and Client Side XSS, Cross-Site Request Forgery, Clickjacking, and Content Sniffing, thereby ensuring robust web application security.
The decision associated with subgoal 1.1.4 and its SA requirements are shown in the following figures.

\begin{figure}[H]
    \centering
    \includegraphics[width=\textwidth]{./assets/subgoal_1.1.4.png}
    \caption{Sub-Goal 1.1.4}
    \label{fig:subgoal1.1.4}
\end{figure}

\begin{table}[H]
\begin{center}
\begin{tabular}{ | m{5cm} | m{5cm}| m{5cm} | } 
  \hline
  \textbf{Level 1 SA requirements} & \textbf{Level 2 SA requirements}  & \textbf{Level 3 SA requirements}  \\ 
  \hline
  HTML, CSS, PHP, Javascript & Managing secure sessions, including authentication, authorization, passwords, and cookies, REST, SOAP and GraphQL services, security in GIT & Understanding how to make a secure web application\\ 
  \hline
  ZAP &  & Recognizing Server Side \& Client Side XSS, Cross-Site Request Forgery, Clickjacking, Content Sniffing\\ 
  \hline
  AFL &  & \\ 
  \hline
  SonarQube, Flawfinder &  & \\ 
  \hline
\end{tabular}
\end{center}
\caption{SA requirements for subgoal 1.1.4}
\end{table}

\newpage
\subsection{Sub-Goal 1.1.5: Demonstrate a solid proficiency in Exploit Development Essentials}
The objective of demonstrating solid proficiency in exploit development essentials is to enable individuals to effectively identify and develop exploits for various security vulnerabilities. This includes foundational knowledge of network protocols, VPNs, and firewalls. Advanced competencies involve understanding ARM-32 and ARM-64 assembly, manipulating registers, stacks, and functions, and analyzing binary files. Ultimately, this comprehensive skill set allows individuals to understand how malicious scripts affect applications, identify flaws in security measures, and leverage exploit frameworks to enhance cybersecurity defenses.
The decision associated with subgoal 1.1.5 and its SA requirements are shown in the following figures.

\begin{figure}[H]
  \centering
  \includegraphics[width=\textwidth]{./assets/subgoal_1.1.5.png}
  \caption{Sub-Goal 1.1.5}
  \label{fig:subgoal1.1.5}
\end{figure}

\begin{table}[H]
    \begin{center}
    \begin{tabular}{ | m{5cm} | m{5cm}| m{5cm} | } 
      \hline
      \textbf{Level 1 SA requirements} & \textbf{Level 2 SA requirements}  & \textbf{Level 3 SA requirements}  \\ 
      \hline
      Network Protocols & Assembly for ARM-32 and ARM-64 & Understanding how a malicious script affects an application\\ 
      \hline
      VPN &  Registers, stacks and functions & Ability to identify flaws in security measures\\ 
      \hline
      Firewalls & Analysis of binary files & Knowledge of exploits frameworks\\  
      \hline
    \end{tabular}
    \end{center}
    \caption{SA requirements for subgoal 1.1.5}
    \end{table}

\newpage
\section{Major Goal 2.1: Ensure course optimization to maximize learning outcomes}
This Major-Goal aims to make the platform more engaging and effective for users. It focuses on two main things: boosting how users interact with the platform and making studying more personalized based on how each user learns.

To achieve this, the platform will track how users use it—what they're interested in, what skills they have, and where they might need help. With this info, the platform can suggest topics for users to review and adjust how it works to match each user's progress and preferences. This will make learning easier and more effective, helping users reach their goals faster.

\subsection{Sub-Goal 2.1.1: Improve student engagement with the platform}
The objective of improving student engagement with the platform focuses on increasing and enhancing student interactions. Key metrics include the frequency of logins, session length, click-through rates on content, return visits, retention rates, and forum activity (questions and answers). Comparative analyses involve user login frequencies versus average rates, material usage percentages, and forum activity levels. Ultimately, the goal is to prevent student dropout by understanding and addressing engagement factors.
The decision associated with subgoal 2.1.1 and its SA requirements are shown in the following figures.
\begin{figure}[H]
    \centering
    \includegraphics[width=\textwidth]{./assets/subgoal_1.2.1.png}
    \caption{Sub-Goal 2.1.1}
    \label{fig:subgoal1.2.1}
\end{figure}

\begin{table}[H]
\begin{center}
\begin{tabular}{ | m{5cm} | m{5cm}| m{5cm} | } 
  \hline
  \textbf{Level 1 SA requirements} & \textbf{Level 2 SA requirements}  & \textbf{Level 3 SA requirements}  \\ 
  \hline
  The frequency of logins & Comparison between user logins onto the platform and the average login rate of other users & Preventing student dropout \\ 
  \hline
  Session length & Percentage of usage of the different kinds of materials provided to the students & \\ 
  \hline
  Click-through Rate (CTR) on Content & Comparison of activity on the forum between users & \\ 
  \hline
  Return Visits and Retention Rates &  & \\ 
  \hline
  Number of answers on the forum &  & \\ 
  \hline
  Number of questions on the forum &  & \\ 
  \hline
  Number of notifications on the forum &  & \\ 
  \hline
  Number of active users  &  & \\ 
  \hline
\end{tabular}
\end{center}
\caption{SA requirements for subgoal 2.1.1}
\end{table}

\newpage
\subsection{Sub-Goal 2.1.2: Optimize student learning through personalization}
The objective of optimizing student learning through personalization is to enhance educational effectiveness by tailoring the learning experience to individual student needs and preferences. This involves assessing student performance through end-of-module assessments and tracking course completion rates to understand overall engagement. Additionally, it requires analyzing trends in student performance over time to identify areas for improvement. Furthermore, the initiative aims to optimize learning by monitoring resource reuse, identifying areas of difficulty and preferred learning styles, and evaluating the acquisition of specific skills. By leveraging these insights, the goal is to create a more personalized educational environment that supports and enhances student learning outcomes effectively.
\begin{figure}[H]
    \centering
    \includegraphics[width=\textwidth]{./assets/subgoal_1.2.2.png}
    \caption{Sub-Goal 2.1.2}
    \label{fig:subgoal1.2.2}
\end{figure}

\begin{table}[H]
\begin{center}
\begin{tabular}{ | m{5cm} | m{5cm}| m{5cm} | } 
  \hline
  \textbf{Level 1 SA requirements} & \textbf{Level 2 SA requirements}  & \textbf{Level 3 SA requirements}  \\ 
  \hline
  End-of-module assessments for each student & Percentage of completed course & Trends in student performance over time \\ 
  \hline
  Number of reuses of a resource & Skill types and areas of difficulty & \\ 
  \hline
  Modules visited & Problem-Solving vs Memory Performance & \\ 
  \hline
  Used material & Student Preferences & \\ 
  \hline
  Skills acquired &  & \\ 
  \hline
\end{tabular}
\end{center}
\caption{SA requirements for subgoal 2.1.2}
\end{table}